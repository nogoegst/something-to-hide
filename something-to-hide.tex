\documentclass{beamer}
%\usepackage[orientation=landscape,size=a4,scale=2.7]{beamerposter}
\usepackage[size=custom,height=21.0,width=33.6,scale=2.7]{beamerposter}
\usepackage[utf8]{inputenc}
\usepackage{default}
\usepackage{graphicx}
\usepackage{calc}
\usepackage{tikz}
\usepackage{fontspec}
\usepackage{color}

\usefonttheme{professionalfonts}
\usefonttheme{serif}

%\setmainfont{Gill Sans Pro Cyrillic Medium}
\setmainfont{P22 Underground CY Pro Demi}

\setbeamertemplate{navigation symbols}{}%remove navigation symbols

\makeatletter
\define@key{beamerframe}{c}[true]{% centered
  \beamer@frametopskip=0pt plus 1fill\relax%
  \beamer@framebottomskip=0pt plus 1fill\relax%
  \beamer@frametopskipautobreak=0pt plus .4\paperheight\relax%
  \beamer@framebottomskipautobreak=0pt plus .6\paperheight\relax%
  \def\beamer@initfirstlineunskip{}%
}
\makeatother

\definecolor{slidebg}{HTML}{314046}
\definecolor{slidefg}{HTML}{DEE2EB}

\definecolor{sgreen}{HTML}{8EB000}
\definecolor{sred}{HTML}{CF4500}


\setbeamercolor{background canvas}{bg=slidebg}
\setbeamercolor{normal text}{fg=slidefg}

\newcommand{\slide}[1]{
	\begin{frame}[c]
	\centering
	{#1}
	\end{frame}
}

\newcommand{\pic}[1]{
	\begin{frame}[c]
	\centering
		\begin{tikzpicture}[remember picture,overlay]
		    \node[at=(current page.center)] {
			\includegraphics[width=\paperwidth]{{#1}}
		    };
		\end{tikzpicture}

	\end{frame}
}



\begin{document}

\pic{pet-media/cryptoparty/global-label-large.png}
%\slide{КриптоПати - не решит массовую слежку}
%\slide{Не все могут/хотят придти на КП}
% call for geeks

%\slide{\textit{Мне нечего скрывать}}
%\slide{Неправда}

\slide{Приватность}
\slide{Право на выбор \\ самовыражения}
\slide{Личное пространство \\ \tiny{лаборатория мыслей \\}}

\slide{Фундаментальное право}
\slide{Пренебрежение \\ личным правом \\ влечет ограничение права других}
\slide{Культура массовой слежки}

%\slide{Добро пожаловать в мир \textit{массовой} слежки\\}
\slide{Контроль}


\slide{Приватность опирается \\ на безопасность}
\slide{Безопасность}
\slide{\tiny{То, что находится между представлением того,
 как работают вещи и как они работают \textit{на самом деле}. \\
}}
\slide{\tiny{Безопасность нельзя \\ купить, скачать или установить} \\ }
\slide{Безопасность - \\ это процесс}

\slide{Для безопасных коммуникаций \\ нет бизнес модели}

\slide{\textit{Интернет}}
\slide{Интернет вошел в нашу жизнь}
\slide{Как передается информация через Интернет?}
\slide{Интернет - \\ сеть сетей}
\slide{Передача \\ из одной сети \\ в другую}
\slide{Через множество посредников}
\pic{pet-media/internet/1.png}
\pic{pet-media/internet/2.png}
\slide{Интернет - \textit{просто} сеть передачи данных}
\slide{Посредниками \\ могут быть \\ \textit{кто-угодно}}
\pic{pet-media/internet/3.png}
\pic{pet-media/internet/4.png}
\pic{pet-media/internet/5.png}
\pic{pet-media/internet/6.png}


\slide{\tiny{Криптография - \\ методы безопасных коммуникаций} \\ }
\slide{Что такое \\ шифрование?}
\slide{Методы превращения полезной информации \\ в бессмыслицу}
\slide{Бессмыслицу, которая "понятна" \textbf{только} вам}
\slide{\tiny{было: cu at cryptoparty \\ стало: 9d23fb0afafa37a57dafa \\}}

%<----
%\slide{\textit{"Шифрование - \\ это сложно"}}
%\slide{Компьютеры - это тоже сложно, но вы ими пользуетесь}
%\slide{пользоваться \\ ≠ \\ разбираться}
%\slide{\textit{"Шифрование - \\ для террористов \\ и военных"}}
%\slide{Так же как и ножи - \\ для убийц и маньяков}
%<----

\slide{\textit{Какое бывает шифрование?}}
\slide{\textbf{Симметричное}}
\slide{Единый ключ \\ на зашифровку \\ и на расшифровку}
\pic{pet-media/crypto/encryption/symmetric/1.png}
\pic{pet-media/crypto/encryption/symmetric/2.png}
\pic{pet-media/crypto/encryption/symmetric/3.png}
\pic{pet-media/crypto/encryption/symmetric/4.png}
\pic{pet-media/crypto/encryption/symmetric/5.png}
\pic{pet-media/crypto/encryption/symmetric/6.png}
\pic{pet-media/crypto/encryption/symmetric/7.png}
\slide{\tiny{Например, симметричный ключ \\ выглядит так: \\
e3594d0ce14fd79425921123d8ec81ea \\
}}
\slide{Например, шифрование паролем}

\slide{Как безопасно передать ключ публично?}
\slide{\textbf{Асимметричное} \\ \tiny{Криптосистема с публичным ключом}}
\slide{\textcolor{sgreen}{Публичный ключ} \\ для зашифровки \\ \textcolor{sred}{Закрытый ключ} \\ для расшифровки}
\pic{pet-media/crypto/encryption/asymmetric/1.png}
\pic{pet-media/crypto/encryption/asymmetric/2.png}
\pic{pet-media/crypto/encryption/asymmetric/3.png}
\pic{pet-media/crypto/encryption/asymmetric/4.png}
\pic{pet-media/crypto/encryption/asymmetric/5.png}
\pic{pet-media/crypto/encryption/asymmetric/6.png}
\pic{pet-media/crypto/encryption/asymmetric/7.png}
\pic{pet-media/crypto/encryption/asymmetric/1.png}
\slide{\tiny{Например, публичный ключ выглядит так: }}
\pic{pet-media/crypto/encryption/asymmetric/gpg-public-key-example.png}
\slide{\tiny{Например, закрытый ключ выглядит так: }}
\pic{pet-media/crypto/encryption/asymmetric/gpg-private-key-example.png}

\slide{\textbf{Сквозное шифрование} \\ \tiny{end-to-end encryption}}
\slide{Шифрование без третьих лиц}
%\slide{Собеседники сами генерируют свои ключи}
%\slide{Содержимое доступно только им}

\slide{\textbf{Forward Secrecy} \\ \tiny{Будущая секретность}}
\slide{Сообщения одноразовые \\ \tiny{Прошлые сообщения не поддаются расшифровке}}

\slide{\textit{Цифровая подпись}}
\slide{\tiny{Только Боб может создать подпись сообщения, \\ которая связана только с этим сообщением. \\ Её можно проверить с помощью публичного ключа Боба.\\}}
\slide{\textit{Создание подписи}}
\pic{pet-media/crypto/signature/signature-create-simplified.png}
\slide{\textit{Проверка подписи}}
\pic{pet-media/crypto/signature/signature-verify-simplified.png}
\slide{\tiny{Например, подписанный документ выглядит так:
}}
\pic{pet-media/crypto/signature/gpg-signature-example.png}


\slide{А что, если замок \\ прислал \textit{не Боб}?}
\slide{Надо убедиться, \\ что это его ключ}
\slide{\textit{Ручная сверка \\ публичных ключей}}
%\slide{Можно попросить кого-то подписать публичный ключ }
\slide{Отпечаток ключа \\ \tiny{Fingerprint}}
\slide{\tiny{Короткий "эквивалент" публичного ключа. \\ Если отпечатки публичных ключей совпадают, значит совпадают и эти ключи. \\ }}
\slide{\tiny{Например, отпечаток ключа выглядит так: }}
\pic{pet-media/crypto/encryption/asymmetric/gpg-fingerprint-example.png}


\slide{Содержание передаваемых нами данных неизвестно}
%metadata gif
\slide{Метаданные \\ \tiny{данные о данных}}
\slide{\tiny{Размер, время/место передачи, \\ получатель/отправитель...\\ }}
%moremoremore
%\slide{\tiny{\textit{"Метаданные мало, что дают,\\ собирайте, сколько хотите" \\ }}}
\slide{Метаданные несут \\ очень большую опасность}
\slide{"Мы убиваем людей \\ на основе метаданных" - \\ \tiny{Майкл Хайден, директор АНБ (1999-2005)}}
\slide{Наиболее интересны}
\slide{Универсальны}
\slide{Не зависят от языка}
\slide{Малый объем}
\slide{Из них можно \\ узнать больше, чем \\ из самих данных}
\slide{\tiny{Любимые места, заболевания, пристрастия, \\ социальные связи, настроение, планы на будущее... \\ }}
\slide{Теперь с обязательным хранением \\ \tiny{по закону - "до трех лет"}}

\slide{\textit{Важность \\свободного ПО}}
\slide{Приватность невозможна без безопастости}
\slide{Безопасность \\ \textbf{и} приватность  \textbf{невозмозможны} \\ без СПО}
\slide{Доступность \\ всех деталей \\ и исходного кода}
\slide{Отсутствие ограничений \\ на использование}
\slide{Проверяемость \\ \tiny{очень сложно встроить лазейки}}
\slide{Независимость}

\slide{\textit{Основные инструменты \\ защиты приватности}}
\slide{\textit{Защита метаданных, сетевой активности и обход цензуры}}

\pic{pet-media/tools/tor/logo/Tor-logo-2011-flat.png}

\slide{The onion router \\ \tiny{Луковичный маршрутизатор}}
\slide{Луковичная маршрутизация}
\slide{Разделение маршрутизации \\и данных \\ \tiny{техническая анонимность\\}}
\slide{Маршрут выбирает пользователь, а не операторы сети}
\slide{"Не существует вездесущего наблюдателя" \\ \tiny{предположение}}

\pic{pet-media/nsa/tor/king_red2.png}

\slide{Как устроен Tor?}
\slide{Без Tor}

\pic{pet-media/tools/tor/howitworks/png/internet/1.png}
\pic{pet-media/tools/tor/howitworks/png/internet/2.png}
\pic{pet-media/tools/tor/howitworks/png/internet/3.png}
\pic{pet-media/tools/tor/howitworks/png/internet/4.png}
\pic{pet-media/tools/tor/howitworks/png/internet/5.png}
\pic{pet-media/tools/tor/howitworks/png/internet/6.png}
\pic{pet-media/tools/tor/howitworks/png/internet/7.png}
\slide{Выход в Интернет \\ через Tor}
\pic{pet-media/tools/tor/howitworks/png/exit/1.png}
\pic{pet-media/tools/tor/howitworks/png/exit/2.png}
\pic{pet-media/tools/tor/howitworks/png/exit/3.png}
\pic{pet-media/tools/tor/howitworks/png/exit/4.png}
\pic{pet-media/tools/tor/howitworks/png/exit/5.png}
\pic{pet-media/tools/tor/howitworks/png/exit/6.png}
\pic{pet-media/tools/tor/howitworks/png/exit/7.png}
\pic{pet-media/tools/tor/howitworks/png/exit/8.png}
\pic{pet-media/tools/tor/howitworks/png/exit/9.png}
\pic{pet-media/tools/tor/howitworks/png/exit/10.png}
\pic{pet-media/tools/tor/howitworks/png/exit/11.png}
\pic{pet-media/tools/tor/howitworks/png/exit/12.png}
\pic{pet-media/tools/tor/howitworks/png/exit/13.png}
\slide{\textit{Кто-то} говорит \textit{с кое-кем}}
\slide{\textit{Выходной релей видит \\ передаваемые данные!}}

\slide{Цель: \\\textit{кто-то} говорит с \textit{кем-то}}
\slide{Луковичные сервисы Tor \\ \tiny{Tor Onion Services}}
\slide{Анонимность как клиента, \\ так и сервера}
\slide{Публичные ключи \\ вместо имен \\ \tiny{Достаточно сгенерировать ключевую пару}}
\slide{\tiny{zkym3uprkoddlxpq.onion}}
\slide{\tiny{facebookcorewwwi.onion}}
\slide{Сквозное шифрование}
\slide{Forward Secrecy}
\slide{Запускаются везде}
\pic{pet-media/tools/tor/howitworks/png/rendezvous/1.png}
\pic{pet-media/tools/tor/howitworks/png/rendezvous/2.png}
\pic{pet-media/tools/tor/howitworks/png/rendezvous/3.png}
\pic{pet-media/tools/tor/howitworks/png/rendezvous/4.png}
\pic{pet-media/tools/tor/howitworks/png/rendezvous/5.png}
\pic{pet-media/tools/tor/howitworks/png/rendezvous/6.png}
\pic{pet-media/tools/tor/howitworks/png/rendezvous/7.png}
\pic{pet-media/tools/tor/howitworks/png/rendezvous/8.png}
\pic{pet-media/tools/tor/howitworks/png/rendezvous/9.png}
\pic{pet-media/tools/tor/howitworks/png/rendezvous/10.png}
\pic{pet-media/tools/tor/howitworks/png/rendezvous/11.png}
\slide{Tor Browser}
\slide{Mozilla Firefox* + tor  \\ \tiny{ * с огромным количеством исправлений \\ для повышения приватности\\}}
\slide{\tiny{Tor Browser \\ выглядит так: \\}}
\pic{pet-media/tools/tor/tor-browser-screenshot.png}
\slide{Мосты \\ \tiny{Релеи Tor, \\ которых нет \\ в публичном \\ состоянии сети \\}}
\slide{Используются как \\ входные релеи \\ в цепочках }
\slide{Подключаемые Транспорты \\ \tiny{Pluggable Transports}}
\slide{Алгоритмы маскировки подключений Tor \\ \tiny{для обхода цензуры}}


\slide{\textit{Защита чатов}}
\slide{OTR \\ \tiny{Off-The-Record Messaging}}
\slide{Почему OTR?}
\slide{Открытый}
\slide{Простой}
\slide{Хорошо проанализирован}
\slide{Отрицание авторства}
\slide{Forward Secrecy}
\pic{pet-media/nsa/no-decrypt/otr-no-decrypt.png}
\slide{Устанавлиается зашифрованый канал}
\slide{Можно сверить отпечаток, чтобы убедится в подлиности}
\pic{pet-media/tools/otr/png/otr-how1.png}
\pic{pet-media/tools/otr/png/otr-how2.png}
\pic{pet-media/tools/otr/png/otr-how3.png}
\pic{pet-media/tools/otr/png/otr-how4.png}
\pic{pet-media/tools/otr/png/otr-how5.png}
\pic{pet-media/tools/otr/png/otr-how6.png}
\slide{\tiny{Сообщение OTR выглядит так: \\
?OTRv2?Y3J5cHRvcGFydHkK
}}
\slide{В текущей версии протокола \\ нет передачи файлов \\ \tiny{Используйте OpenPGP/OnionShare}}
\slide{Предпочтительный вариант \\ использования OTR}
\slide{Протокол XMPP}
\slide{Децентрализованный}
\slide{Похож на эл. почту}
\slide{\tiny{cryptoparty@jabber.ccc.de}}
\slide{Можно зарегистрироваться \\ на любом сервере}
\slide{Можно связываться \\ с пользователями на любом сервере}


\slide{\textit{Защита файлов/почты}}
\slide{OpenPGP}
\slide{Система \\ шифрования/подписи \\ данных}
\slide{Почему OpenPGP?}
\slide{30 лет \\ в использовании }
\slide{Многочисленные аудиты}
\slide{Стандарт Интернета}
\pic{pet-media/nsa/no-decrypt/pgp_no_decrypt.png}
\pic{pet-media/tools/pgp/gnupg/logo/Gnupg-logo-vertical.png}
\slide{\tiny{Публичный ключ выглядит так:
}}
\pic{pet-media/crypto/encryption/asymmetric/gpg-public-key-example.png}
\slide{\tiny{Например, зашифрованное сообщение \\ выглядит так:
}}
\pic{pet-media/crypto/encryption/gpg-encryption-example.png}


\slide{\textit{Защита работы на компьютере}}
\slide{Tails}
\slide{Ориентированный на приватность \\ вариант GNU\textbackslash Linux}
\slide{Запускается с флешки}
\slide{Все соединения через \\ Tor или I2P}
\slide{TorBrowser, OTR, GPG \\ и все, что нужно \\ для работы}
\pic{pet-media/tools/tails/tails.png}

\slide{Ссылки}
\slide{Пароль WiFi: \\ password}
\slide{\tiny{Tor \\ https://torproject.org/ \\ Orbot (Android) \\}}
\slide{\tiny{F-Droid \\ https://f-droid.org/}}
\slide{\tiny{OTR \\ https://otr.cypherpunks.ca \\ Pidgin/Adium/CoyIM \\ (Linux, Windows, macOS) \\ Conversations (Android) \\ ChatSecure (iOS, Android) \\}}
\slide{\tiny{GnuPG \\ https://gnupg.org/ \\ GnuPG+Thunderbird+Enigmail (Desktop) \\OpenKeyChain+K9-Mail (Android) \\ }}
\slide{\tiny{Tails \\ https://tails.boum.org/ \\}}


\end{document}
